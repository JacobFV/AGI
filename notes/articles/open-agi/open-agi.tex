%% arara directives
% arara: xelatex
% arara: bibtex
% arara: xelatex
% arara: xelatex

%\documentclass{article} % One-column default
\documentclass[twocolumn, switch]{article} % Method A for two-column formatting

\usepackage{preprint}

%% Math packages
\usepackage{amsmath, amsthm, amssymb, amsfonts}

%% Bibliography options
\usepackage[numbers,square]{natbib}
\bibliographystyle{unsrtnat}
%\usepackage{natbib}
%\bibliographystyle{Geology}

%% General packages
\usepackage[utf8]{inputenc}	% allow utf-8 input
\usepackage[T1]{fontenc}	% use 8-bit T1 fonts
\usepackage{xcolor}		% colors for hyperlinks
\usepackage[colorlinks = true,
            linkcolor = purple,
            urlcolor  = blue,
            citecolor = cyan,
            anchorcolor = black]{hyperref}	% Color links to references, figures, etc.
\usepackage{booktabs} 		% professional-quality tables
\usepackage{nicefrac}		% compact symbols for 1/2, etc.
\usepackage{microtype}		% microtypography
\usepackage{lineno}		% Line numbers
\usepackage{float}			% Allows for figures within multicol
%\usepackage{multicol}		% Multiple columns (Method B)

\usepackage{lipsum}		%  Filler text

 %% Special figure caption options
\usepackage{newfloat}
\DeclareFloatingEnvironment[name={Supplementary Figure}]{suppfigure}
\usepackage{sidecap}
\sidecaptionvpos{figure}{c}

% Section title spacing  options
\usepackage{titlesec}
\titlespacing\section{0pt}{12pt plus 3pt minus 3pt}{1pt plus 1pt minus 1pt}
\titlespacing\subsection{0pt}{10pt plus 3pt minus 3pt}{1pt plus 1pt minus 1pt}
\titlespacing\subsubsection{0pt}{8pt plus 3pt minus 3pt}{1pt plus 1pt minus 1pt}


%%%%%%%%%%%%%%%%   Title   %%%%%%%%%%%%%%%%
\title{The Road to AGI}

% Add watermark with submission status
\usepackage{xwatermark}
% Left watermark
\newwatermark[firstpage,color=gray!60,angle=90,scale=0.32, xpos=-4.05in,ypos=0]{\href{https://doi.org/}{\color{gray}{Publication doi}}}
% Right watermark
\newwatermark[firstpage,color=gray!60,angle=90,scale=0.32, xpos=3.9in,ypos=0]{\href{https://doi.org/}{\color{gray}{Preprint doi}}}

%%%%%%%%%%%%%%%  Author list  %%%%%%%%%%%%%%%
\author{
  Jacob F.~Valdez \\
  % Department of Computer Science\\
  % Cranberry-Lemon University\\
  % Pittsburgh, PA 15213 \\
  \texttt{jacobfv@msn.com}
}

%%%%%%%%%%%%%%    Front matter    %%%%%%%%%%%%%%
\begin{document}

\twocolumn[ % Method A for two-column formatting
  \begin{@twocolumnfalse} % Method A for two-column formatting
  
\maketitle

\begin{abstract}
\end{abstract}
%\keywords{First keyword \and Second keyword \and More} % (optional)
\vspace{0.35cm}

  \end{@twocolumnfalse} % Method A for two-column formatting
] % Method A for two-column formatting

%\begin{multicols}{2} % Method B for two-column formatting (doesn't play well with line numbers), comment out if using method A


%%%%%%%%%%%%%%%  Main text   %%%%%%%%%%%%%%%
\linenumbers

\section{Generality - a key factor of intelligence}

Different from lifeless information and the information spaces that can be constructed therefrom, intelligence space spans the bredth and depth of all intelligence. Differing domains of information may rest on similar fundemental principles of intelligence. For instance, rule-based, symbolic, and other narrow AI systems sit on the surface of this space only occupying small niches, while living systems such as the biosphere, ecosystems, communities, populations, organisms, cells, and viroids stretch across vast manifolds of intellectual space. One may even imagine clusters of similar DNA's as composing intelligences in themselves. Just as the ecosystem lives through its organisms and the multicellular organism by its cells, categories of DNA react intelligently to their internal and external environment by mutation and propagation. The genetic intelligences today are highly optimized for their current respective niches and highly sensitive to adaptation.

Taking lessons toward AGI, artificially engineered intelligence becomes more general each day as it pervades industry sectors, application domains, and information spaces. Rather than an abrupt transition towards independance, narrow AI follows a smooth, continuous path towards generality. The question is not: ``is it artificial general intelligence?'' but ``how general is the artificial intelligence?'' The road to AGI is not a journey toward some end where it can 'finally' learn for itself and humans are no longer needed, but an never-ending exploration across the breadths and into the depths of intellectual space.

The scientific community's steps along this path have seen fine progress. More computing resources have long ended the AI Winter. Now neural networks are billions of parameters deep, heterogenous and multimodal machine learning and symbolic logic integrations outperform experts, and natural language and image generation systems produce content that is ``FIND QUOTE LIKE DANGEROUSLY DECEPTIVE''. More importantly, along the way, we are mining principles of intelligence that can be applied as if a ``lever of intelligence`` simply by scaling up parameters and infrastructure. 

\section{Openness}

With growth however comes increasing responsibility to meet the demands of its application effectively. AI

%%%%%%%%%%%% Supplementary Methods %%%%%%%%%%%%
%\footnotesize
%\section*{Methods}

%%%%%%%%%%%%% Acknowledgements %%%%%%%%%%%%%
%\footnotesize
%\section*{Acknowledgements}

%%%%%%%%%%%%%%   Bibliography   %%%%%%%%%%%%%%
\normalsize
\bibliography{references}

%%%%%%%%%%%%  Supplementary Figures  %%%%%%%%%%%%
%\clearpage

%%%%%%%%%%%%%%%%   End   %%%%%%%%%%%%%%%%
%\end{multicols}  % Method B for two-column formatting (doesn't play well with line numbers), comment out if using method A
\end{document}
